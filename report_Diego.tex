%%%%%%%%%%%%%%%%%%%%%%%%%%%%%%%%%%%%%%%%%
% Thin Sectioned Essay
% LaTeX Template
% Version 1.0 (3/8/13)
%
% This template has been downloaded from:
% http://www.LaTeXTemplates.com
%
% Original Author:
% Nicolas Diaz (nsdiaz@uc.cl) with extensive modifications by:
% Vel (vel@latextemplates.com)
%
% License:
% CC BY-NC-SA 3.0 (http://creativecommons.org/licenses/by-nc-sa/3.0/)
%
%%%%%%%%%%%%%%%%%%%%%%%%%%%%%%%%%%%%%%%%%

%----------------------------------------------------------------------------------------
%	PACKAGES AND OTHER DOCUMENT CONFIGURATIONS
%----------------------------------------------------------------------------------------

\documentclass[a4paper, 11pt]{article} % Font size (can be 10pt, 11pt or 12pt) and paper size (remove a4paper for US letter paper)

\usepackage[protrusion=true,expansion=true]{microtype} % Better typography
\usepackage{graphicx} % Required for including pictures
\usepackage{wrapfig} % Allows in-line images

\usepackage{mathpazo} % Use the Palatino font
\usepackage[T1]{fontenc} % Required for accented characters
\linespread{1.05} % Change line spacing here, Palatino benefits from a slight increase by default

\makeatletter
\renewcommand\@biblabel[1]{\textbf{#1.}} % Change the square brackets for each bibliography item from '[1]' to '1.'
\renewcommand{\@listI}{\itemsep=0pt} % Reduce the space between items in the itemize and enumerate environments and the bibliography

\renewcommand{\maketitle}{ % Customize the title - do not edit title and author name here, see the TITLE block below
\begin{flushright} % Right align
{\LARGE\@title} % Increase the font size of the title

\vspace{50pt} % Some vertical space between the title and author name

{\large\@author} % Author name
\\\@date % Date

\vspace{40pt} % Some vertical space between the author block and abstract
\end{flushright}
}

%----------------------------------------------------------------------------------------
%	TITLE
%----------------------------------------------------------------------------------------

\title{\textbf{The Stallion}\\ % Title
Group 9 Project Report\\DD2425} % Subtitle

\author{\textsc{
Tobias Andersson\\
Max Losch\\
Diego Martinez Marrodan\\
Tiago Sebastiao\\
Lan Wang} % Author
\\{\textit{Royal Institute of Technology}}
} % Institution

\date{\today} % Date

%----------------------------------------------------------------------------------------

\begin{document}

\maketitle % Print the title section

%----------------------------------------------------------------------------------------
%	ABSTRACT
%----------------------------------------------------------------------------------------

%\renewcommand{\abstractname}{Summary} % Uncomment to change the name of the abstract to something else

\begin{abstract}
TODO
\end{abstract}

\vspace{30pt} % Some vertical space between the abstract and first section

%----------------------------------------------------------------------------------------
%	ESSAY BODY
%----------------------------------------------------------------------------------------

\section*{Lessons Learned}

\hspace{10 mm}I learned many things about robotics and also about projects more oriented to programming hardware that will work on a physical environment (the projects I did before were software oriented, and it's easier to make them work right since you don't have to care about physical variances or noise). 
I improved some of my programming abilities, especially in C++ (a programming language that I had never used, despite I used C before).
And also I learned some basics about control (but not computer vision since I wasn't in charge of that part of the project. TO OMIT) (DIEGO)

%------------------------------------------------

\section*{Navigation/Mapping (Occupancy grid)}
\hspace{10 mm}The occupancy grid is an adaptation from the probability grid in order to represent the map in rviz.
It uses a one-dimensional array of size \textit{width*height} where width is the width of the map represented and height is its height.
It contains values 0, 100 or others. 
The color representation of those values in the map visualization is white, black and grey respectively.
0 represents that the cell is unoccupied; 100 that it is occupied; and anything else is unknown.


%------------------------------------------------

\section*{Controlling (wall following, imu)}
\hspace{10 mm}The wall following controller uses the IR sensors to measure the distances from the robot to the walls, and works as follows: if both walls are close to the robot (distance < 0.4 m), then keep the robot aligned to the walls in the center.
If the left wall is close but the right one is far, follow the left wall.
And if the right wall is close but the left one is far, follow the right wall.
It also can be activated/deactivated when needed (e.g. when turning) sending a signal.\\

\hspace{10 mm}The IMU is used to detect collisions. 
To do so, we look for peaks in the \textit{/imu/data/linear\_acceleration/y} topic, what would indicate a front collision of the robot with the wall.
To detect these peaks, we apply a high pass filter and then we send a signal to the brain when this filtered signal exceeds a threshold.

%------------------------------------------------

%\section*{Conclusion}

%Herpa derpa derp.

%\begin{enumerate}
%\item First numbered list item
%\item Second numbered list item
%\end{enumerate}

%Donec luctus tincidunt mauris, non ultrices ligula %aliquam id. Sed varius, magna a faucibus congue, arcu %tellus pellentesque nisl, vel laoreet magna eros et %magna. 

%----------------------------------------------------------------------------------------
%	BIBLIOGRAPHY
%----------------------------------------------------------------------------------------

\bibliographystyle{unsrt}

\bibliography{sample}

%----------------------------------------------------------------------------------------

\end{document}