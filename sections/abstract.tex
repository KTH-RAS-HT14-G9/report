\begin{abstract}

For this project, we were asked for doing a robot able to go through a maze and recognize some objects.
Also, it has to be able to localize its position in the maze and visit some objects previously recognized.\\

To achieve this, we went through some milestones that helped us not to be left to an open problem which would be harder to tackle, and to the teachers to evaluate us.
Also, we had a competition for comparing our solution with the ones given by the rest of the students of our class.
After having a little contact with some vision tasks in milestone 0, we had to implement some controllers to make the robot follow a wall for milestone 1.
After that, we had to build a map that would allow the robot to localize itself inside the maze, and to start recognizing some objects according to their shape and their colour.
For this map, we used an occupancy grid.
Finally, we had to find the shortest path in which we visit every recognized object.
For this purpose, we implemented a topological map and we used the brute force version of the TSP algorithm to find this path.\\

We had some problems along the way, but we managed to solve almost every single one.
Nevertheless, there were some of them that we would have liked to solve if we had more time, like doing the vision more robust, use local obstacle avoidance, have a better localization of the robot inside the map, or use more the information from the grid map to get better results.

\end{abstract}