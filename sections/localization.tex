\section{Localization}

Observing the created map given the regular odometry, indicated problems with the heading of the robot.
A constant drift was observable while turning and because angular errors accumulate during turns,
the map got in danger of falling into an unusable state.
Completing a complete round trip in the maze, would inevitably lead to overlapping walls and free space,
making navigation hard or impossible.\\

The common and generic approach for localization would be to use some kind of Bayesian Filter,
which corrects the odometry by receiving measurements from the environment.
Due to very little time until the third milestone, which required to have a basic localization implemented,
it was decided to not spend time on any sophisticated solutions like Kalman or Particle Filter.
Instead, systematic errors were minimized and the prior knowledge of rectangular walls was exploited to use IR readings to correct
the heading of the robot to well defined events.

To minimize systematic errors, too approaches were implemented.
The first and easiest approach is to fix the error in the wheel distance (in the following called base distance) by doing manual hill climbing.
This can be performed when letting the robot turn for at least 3 complete revolutions and roughly determine if the turning overshooted or undershooted.
Based on this information, the base distance can be adjusted accordingly.
The process was repeated until there was no significant drift left.

Furthermore, as stated in \cite{OdomCorr}, a systematic error can be non equal wheel diameters.
Due to different weight distributions and that the wheels are made of soft plastic, 
the wheel diameter can differ in the magnitude of one millimeter. 
Although this is hardly noticeable in the length of the maze it was decided to give it a try.
As described in \cite{OdomCorr} five clockwise and five counter-clockwise runs were performed in a 4mx4m region.
The results were somewhat disappointing though. 
The calculated difference in the wheel diameters turned out to be too large which made the odometry worse than before.
The unexpected result is probably an issue of another systematic error in performing the measurement of the actual position.
It would be advised to perform more than 5 runs to reduce these errors.

Having the angular drift minimized was enough to achieve satisfying results.
To achieve an overall better localization, it was decided on using the IR sensors and also segmented planes from the point cloud, to estimate the position of the robot more accurately.
The IR sensors were used to correct the heading of the robot, by calculating the actual heading from the two side sensors (see equation \ref{eq:ir_theta}).
A precondition that has to be valid to correct $\Theta$ this way, is that the wall next to the robot has to be planar.
This was approached, by only correcting theta after a turn has been performed and only if the calculated theta did not significantly differ from the current estimated odometry.

\begin{equation}
\Theta = tan^{-1}\left( \frac{ d_{ir_{front}} - d_{ir_{back}} }{dist(ir_{front}, ir_{back})} \right)
\label{eq:ir_theta}
\end{equation}

To also correct the position and not only the heading, information about walls were combined with the built map.
Splitting the position in a lateral (y-axis) and a longitudinal (x-axis) part, enables to simplify the correction into two steps.
The longitudinal correction can be solved by casting a ray from robot to the forward facing wall and comparing it with casting a ray in the constructed grid map.
As the point cloud gives quite accurate results when close and while standing, the x position of the robot can be easily corrected.
A very similar approach can be used to correct the lateral direction. 
By comparing the result of a physical IR raycast of the side sensors with a raycast in the grid map, one can derive the correct y position in the map.
Both lateral and longitudinal correction only works, when a map is given and not updated.
For exploration, only the heading correction can be used.\\

The final solution did not use the lateral and the longitudinal correction.
Although implemented it could not be tested properly due to changed priorities.