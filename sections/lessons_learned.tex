\section{Lessons learned}

%% Lan --->
For the object detection and classification.
We had problems regarding shape recognition since plane fitting with RANSAC didn't work well with a noisy camera. 
Instead of point cloud,image based recognition with OpenCV might worth a try. 
Contours of the object can be approximated or detected with line or corner detector, and different shapes should have different characters regarding contours.
Or a Bayes Classifier can be tried as well. \\

\setlength{\parindent}{0pt}A Topological map has been implemented, but it took lots of time to implement and debug.
Maybe it's not really necessary.
Navigation can be implemented based on the occupancy grid map, the odometry, as well as the 'have seen' map. 
So that the robot does not get confused by the unnessesary nodes it dropped, such as explore some unknow direction which does not actually exits.\\

\setlength{\parindent}{0pt}More protection should be implemented for the robot crash.
The IMU seems to be a good choise to get information about from which direction the robot crashes.
Now we only use y axis data, and the robot only goes backwards when the IMU detect crash, since it doesn't know to which direction the obstacle is.
But it can not always solve problems.
Sometimes when the robot crashes into a wall three times in a row, the brain will be dead.
If we read both x and y axis IMU data, we could know to which direction to turn in order to avoid the obstacle.

%% <---